zsim is a fast x86-\/64 simulator. It was originally written to evaluate Z\-Cache (Sanchez and Kozyrakis, M\-I\-C\-R\-O-\/44, Dec 2010), hence the name, but it has since outgrown its purpose. zsim's main goals are to be fast, simple, and accurate, with a focus on simulating memory hierarchies and large, heterogeneous systems. It is parallel and uses D\-B\-T extensively, resulting in speeds of hundreds of millions of instructions/second in a modern multicore host. Unlike conventional simulators, zsim is organized to scale well (almost linearly) with simulated core count.

You can find more details about zsim in our I\-S\-C\-A 2013 paper\-: \href{http://people.csail.mit.edu/sanchez/papers/2013.zsim.isca.pdf}{\tt http\-://people.\-csail.\-mit.\-edu/sanchez/papers/2013.\-zsim.\-isca.\-pdf}.

\subsection*{License \& Copyright }

zsim is free software; you can redistribute it and/or modify it under the terms of the G\-N\-U General Public License as published by the Free Software Foundation, version 2.

zsim was originally written by Daniel Sanchez at Stanford University, and per Stanford University policy, the copyright of this original code remains with Stanford (specifically, the Board of Trustees of Leland Stanford Junior University). Since then, zsim has been substantially modified and enhanced at M\-I\-T by Daniel Sanchez, Nathan Beckmann, and Harshad Kasture. zsim also incorporates contributions on main memory performance models from Krishna Malladi, Makoto Takami, and Kenta Yasufuku.

zsim was also modified and enhanced while Daniel Sanchez was an intern at Google. Google graciously agreed to share these modifications under a G\-P\-Lv2 license. This code is (C) 2011 Google Inc. Files containing code developed at Google have a different license header with the correct copyright attribution.

Additionally, if you use this software in your research, we request that you reference the zsim paper (\char`\"{}\-Z\-Sim\-: Fast and Accurate Microarchitectural
\-Simulation of Thousand-\/\-Core Systems\char`\"{}, Sanchez and Kozyrakis, I\-S\-C\-A-\/40, June 2013) as the source of the simulator in any publications that use this software, and that you send us a citation of your work.

\subsection*{Setup }

External dependencies\-: {\ttfamily gcc $>$=4.\-6, pin, scons, libconfig, libhdf5, libelfg0}

{\bfseries Natively\-:} If you use a relatively recent Linux distribution\-:


\begin{DoxyEnumerate}
\item Clone a fresh copy of the git zsim repository ({\ttfamily git clone $<$path to zsim repo$>$}).
\item Download Pin, \href{http://www.pintool.org}{\tt http\-://www.\-pintool.\-org} . Tested with Pin 2.\-8+ on an x86-\/64 architecture. Compiler flags are set up for Pin 2.\-9 on x86-\/64. To get flags for other versions, examine the Pin makefile or derive from sample pintools. Set the P\-I\-N\-P\-A\-T\-H environment variable to Pin's base directory.

N\-O\-T\-E\-: Linux 3.\-0+ systems require Pin 2.\-10+, just because Pin does a kernel version check that 3.\-0 fails. Use Pin 2.\-12 with Sandy/\-Ivy Bridge systems, earlier Pin versions have strange performance regressions on this machine (extremely low I\-P\-C).
\item zsim requires some additional libraries. If they are not installed in your system, you will need to download and build them\-:

3.\-1 libconfig, \href{http://www.hyperrealm.com/libconfig}{\tt http\-://www.\-hyperrealm.\-com/libconfig}. You may use the system's package if it's recent enough, or build your own. To install locally, untar, run {\ttfamily ./configure -\/-\/prefix=$<$libconfig install path$>$ \&\& make install}. Then define the env var {\ttfamily L\-I\-B\-C\-O\-N\-F\-I\-G\-P\-A\-T\-H=$<$libconfig install path$>$}.

3.\-2 libhdf5, \href{http://www.hdfgroup.org}{\tt http\-://www.\-hdfgroup.\-org} (v1.\-8.\-4 path 1 or higher), and libelfg0. The S\-Construct file assumes these are installed in the system.

3.\-3 (O\-P\-T\-I\-O\-N\-A\-L) polarssl (currently used just for their S\-H\-A-\/1 hash function), \href{http://www.polarssl.org}{\tt http\-://www.\-polarssl.\-org} Install locally as in 3.\-1 and define the env var {\ttfamily P\-O\-L\-A\-R\-S\-S\-L\-P\-A\-T\-H=$<$polarssl install path$>$}.

N\-O\-T\-E\-: You may need to add {\ttfamily -\/f\-P\-I\-C} to the Makefile's C(P\-P/\-X\-X)F\-L\-A\-G\-S depending on the version.

3.\-4 (O\-P\-T\-I\-O\-N\-A\-L) D\-R\-A\-M\-Sim2 for main memory simulation. Build locally and define the env var D\-R\-A\-M\-S\-I\-M\-P\-A\-T\-H as in 3.\-1 and 3.\-3.
\item In some distributions you may need to make minor changes to the host configuration to support large shmem segments and ptrace. See the notes below for more details.
\item Compile zsim\-: {\ttfamily scons -\/j16}
\item Launch a test run\-: {\ttfamily ./build/opt/zsim tests/simple.\-cfg}
\end{DoxyEnumerate}

For more compilation options, run scons --help. You can build debug, optimized and release variants of the simulator (--d, --o, --r options). Optimized (opt) is the default. You can build profile-\/guided optimized (P\-G\-O) versions of the code with --p. These improve simulation performance with O\-O\-O cores by about 30\%.

N\-O\-T\-E\-: zsim uses C++11 features available in {\ttfamily gcc $>$=4.\-6} (such as range-\/based for loops, strictly typed enums, lambdas, and type inference). Older version of gcc will not work. zsim can also be built with {\ttfamily icc} (see the {\ttfamily S\-Construct} file).

{\bfseries Using a virtual machine\-:} If you use another O\-S, can't make system-\/wide configuration changes, or just want to test zsim without modifying your system, you can run zsim on a Linux V\-M. We have included a vagrant configuration file (\href{http://vagrantup.com}{\tt http\-://vagrantup.\-com}) that will provision an Ubuntu 12.\-04 V\-M to run zsim. You can also follow this Vagrantfile to figure out how to setup zsim on an Ubuntu system. Note that {\bfseries zsim will be much slower on a V\-M} because it relies on fast context-\/switching, so we don't recommend this for purposes other than testing and development. Assuming you have vagrant installed ({\ttfamily sudo apt-\/get install vagrant} on Ubuntu or Debian), follow these\-:


\begin{DoxyEnumerate}
\item Copy {\ttfamily misc/\-Vagrantfile} from the repo into an empty folder.
\item Run {\ttfamily vagrant up} to set up the base V\-M and install all dependencies.
\item S\-S\-H into the V\-M with {\ttfamily vagrant ssh}.
\item Inside the V\-M, you can clone the zsim repo, and build it and use it as usual (steps 1, 5, and 6 above).
\end{DoxyEnumerate}

\subsection*{Notes }

{\bfseries Accuracy \& validation\-:} While we have validated zsim against a real system, you should be aware that we sometimes sacrifice some accuracy for speed and simplicity. The I\-S\-C\-A 2013 paper details the possible sources of inaccuracy. Despite our validation efforts, if you are using zsim with workloads or architectures that are significantly different from ours, you should not blindly trust these results. Also, zsim can be configured with varying degrees of accuracy, which may be O\-K in some cases but not others (e.\-g., longer bound phases to reduce overheads are often O\-K if your application has little communication, but not with fine-\/grained parallelism and synchronization). Finally, in some cases you will need to modify the code to model what you want, and for some purposes, zsim is just not the right tool. In any case, we strongly recommend validating your baseline configuration and workloads against a real machine.

In addition to the results in the zsim paper, \href{http://zsim.csail.mit.edu/validation}{\tt http\-://zsim.\-csail.\-mit.\-edu/validation} has updated validation results.

{\bfseries Memory Management\-:} zsim can simulate multiple processes, which introduces some complexities in memory management. Each Pin process uses Sys\-V I\-P\-C shared memory to communicate through a global heap. Be aware that Pin processes have a global and a process-\/local heap, and all simulator objects should be allocated in the global heap. A global heap allocator is implemented (galloc.\-c and g\-\_\-heap folder) using Doug Lea's malloc. The global heap allocator functions are as the usual ones, with the gm\-\_\- prefix (e.\-g. gm\-\_\-malloc, gm\-\_\-calloc, gm\-\_\-free). Objects can be allocated in the global heap automatically by making them inherit from \hyperlink{classGlobAlloc}{Glob\-Alloc}, which redefines the new and delete operators. S\-T\-L classes use their own internal allocators, so they cannot be members of globally visible objects. To ease this, the g\-\_\-stl folder has template specializations of commonly used S\-T\-L classes that are changed to use our own S\-T\-L-\/compliant allocator that allocates from the global heap. Use these classes as drop-\/in replacements when you need a globally visible S\-T\-L class, e.\-g. substitute std\-::vector with \hyperlink{classg__vector}{g\-\_\-vector}, etc.

{\bfseries Harness\-:} While most of zsim is implemented as a pintool ({\ttfamily libzsim.\-so}), a harness process ({\ttfamily zsim}) is used to control the simulation\-: set up the shared memory segment, launch pin processes, check for deadlock, and ensure termination of the whole process tree when it is killed. In prior revisions of the simulator, you could launch the pintool directly, but now you should use the harness.

{\bfseries Transparency \& I/\-O\-:} To maintain transparency w.\-r.\-t instrumented applications, zsim does all logging through info/warn/panic methods. With the sim.\-log\-To\-File option, these dump to per-\/process log files instead of the console. {\itshape You should never use cout/cerr or printf in simulator code} --- simple applications will work, but more complex setups, e.\-g., anything that uses pipes, will break.

{\bfseries Interfacing with applications\-:} You can use special instruction sequences to control the simulation from the application (e.\-g., fast-\/forward to the region you want to simulate). {\ttfamily misc/hooks} has wrappers for C/\-C++, Fortran, and Java, and extending this to other languages should be easy.

{\bfseries Host Configuration\-:} The system configuration may need some tweaks to support zsim. First, it needs to allow for large shared memory segments. Second, for Pin to work, it must allow a process to attach to any other from the user, not just to a child. Use sysctl to ensure that {\ttfamily kernel.\-shmmax=1073741824} (or larger) and {\ttfamily kernel.\-yama.\-ptrace\-\_\-scope=0}. zsim has mainly been used in Ubuntu 11.\-10, 12.\-04, 12.\-10, 13.\-04, and 13.\-10, but it should work in other Linux distributions. Using it in O\-Ss other than Linux (e.\-g,, O\-S X, Windows) will be non-\/trivial, since the user-\/level virtualization subsystem has deep ties into the Linux syscall interface.

{\bfseries Stats\-:} The simulator outputs periodic, eventual and end-\/of-\/sim stats files. Stats can be output in both H\-D\-F5 and plain text. Read the R\-E\-A\-D\-M\-E.\-stats file and the associated scripts repository to see how to use these stats.

{\bfseries Configuration \& Getting Started\-:} A detailed use guide is out of the scope of this R\-E\-A\-D\-M\-E, because the simulator options change fairly often. In general, {\itshape the documentation is the source code}. You should be willing to occasionally read the source code to see how different zsim features work. To get familiar with the way to configure the simulator, the following three steps usually work well when getting started\-:


\begin{DoxyEnumerate}
\item Check the examples in the {\ttfamily tests/} folder, play around with the settings, and launch a few runs. \hyperlink{classConfig}{Config} files have three sections, sys (configures the simulated system, e.\-g., core and cache parameters), sim (configures simulation parameters, e.\-g., how frequent are periodic stats output, phase length, etc.), and process\{0, 1, 2, ...\} entries (what processes to run).
\item Most parameters have implicit defaults. zsim produces an out.\-cfg file that includes all the default choices (and we recommend that your analysis scripts automatically parse this file to check that what you are simulating makes sense). Inspecting the out.\-cfg file reveals more configuration options to play with, as well as their defaults.
\item Finally, check the source code for more info on options. The whole system is configured in the init.\-cpp (sys and sim sections) and process\-\_\-tree.\-cpp (process\-X sections) files, so there is no need to grok the whole simulator source to find out all the configuration options.
\end{DoxyEnumerate}

{\bfseries Hacking \& Style Guidelines\-:} zsim is mostly consistent with Google's C++ style guide. You can use cpplint.\-py to check rule violations. We depart from these guidelines on a couple of aspects\-:


\begin{DoxyItemize}
\item 4-\/space indentation instead of 2 spaces
\item 120-\/character lines instead of 80-\/char (though you'll see a clear disregard for strict line length limits every now and then)
\end{DoxyItemize}

You can use cpplint.\-py (included in misc/ with slight modifications) to check your changes. misc/ also has a script to tidy includes, which should be in alphabetical order within each type (own, system, and project headers).

vim will indent the code just fine with the following options\-: {\ttfamily set cindent shiftwidth=4 expandtab smarttab}

Finally, as Google's style guidelines say, please be consistent with the current style used elsewhere. For example, the parts of code that deal with Pin follow a style consistent with pintools.

Happy hacking, and hope you find zsim useful! 